\section{Second order cone programming}

A convex program is an optimization problem of the form
\begin{equation}
    \begin{aligned}
    & \text{minimize} & & f(\Landmark) \\
    & \text{subject to} & & h_i(\Landmark) \geq 0 ~~~~~~ i = 1,\ldots,N
  \end{aligned}
\end{equation}
where $f$ is a convex function and $\{\Landmark : h_i(\Landmark) \geq 0 ~\forall i\}$
is a convex set. All convex programs can be solved \cite{boyd}, but
certain problem classes permit particularly efficient solutions. One
such class is the second--order cone program (SOCP), which has the
form
\begin{equation}
  \begin{aligned}
    & \text{minimize} & & \vec{f}^T \Landmark & \\
    & \text{subject to} & & 
    \|A_i\Landmark+\vec{b}_i\|_2 \leq \vec{c}_i^T\Landmark + d_i & i =
    1,\ldots,N \\
    & & & G_j\Landmark = \vec{h}_j & j = 1,\ldots,M
  \end{aligned}
  \label{generic-socp}
\end{equation}
SOCP problems can be solved efficiently using primal--dual interior
point methods \cite{boyd} and several off--the--shelf solvers are
available \cite{mosek,cvxopt,cvx}.

\section{Problem Statement}

We wish to estimate the trajectory of a mobile device given three
types of sensor measurements: accelerometer readings, gyro readings,
and image features. These quantities will be estimated with respect to
an arbitrary (not necessarily gravity--aligned) frame of reference we
will call the world frame. We now describe models for each sensor.

Accelerometer measurements are sampled according to
\begin{equation}
  \Accel = R (\LinearAccel + \Gravity) + \Bias + \AccelNoise
\end{equation}
where $R$ is the orientation of the device in the world frame,
$\LinearAccel$ is the acceleration of the device in the world frame,
$\Gravity$ is the gravity vector in the world frame, $\Bias$ is an
additive bias term, and $\AccelNoise$ is a Gaussian white noise term
with covariance $\AccelCov$. Whereas we assume that the bias $\Bias$
is fixed for all sensor measurements \footnotemark, the noise term is
$\AccelNoise$ is sampled independently for each measurement.

\footnotetext{this is a reasonable assumption for MEMS sensors
as found on modern cell phones for trajectories of up to several minutes}

Gyro measurements are sampled according to
\begin{equation}
  \Gyro = R \AngularVelocity + \GyroBias + \GyroNoise
\end{equation}
where $\AngularVelocity$ is the angular velocity of the device in the
world frame, $\GyroBias$ is an additive bias, and $\GyroNoise$ is a
Gaussian white noise term with covariance $\GyroCov$.

Finally, image measurements are sampled according to
\begin{equation}
  \Feature = \Pr(K R (\Landmark - \Position)) + \FeatureNoise
\end{equation}
where $R$ and $p$ are respectively the orientation and position of the
device in the world frame at the time the feature was, $K$ is the
camera intrinsics, $\Landmark$ is the world coordinates of the
landmark giving rise to the feature, and $\FeatureNoise$ is a Gaussian
white noise term with covariance $\FeatureCov$. $\Pr$ is the
projection function,
\begin{equation}
  \Pr\Bigl(\begin{bmatrix}x\\y\\z\end{bmatrix}\Bigr) =
    \frac{1}{z}\begin{bmatrix}x\\y\end{bmatrix}
\end{equation}

\subsection{Trajectory representation}

We choose to parametrize the trajectory of the device as a cubic
B--spline, as described in \cite{furgale,lovegrove}. Given
control points $\Control_1,\cdots,\Control_K$, $\Control_i \in
\Reals$, the position of the device at time $t$ is,
\begin{equation}
  \Position = \sum \SplineCoef_i(t) \Control_i
\end{equation}
where $\SplineCoef_1,\cdots,\SplineCoef_K$ are the spline basis
functions, which depend only on time and can be computed using the de
Boor--Cox formula \cite{deboor1978}. For brevity, we concatenate the
control points into a vector
$\Controls=(\Controls_1,\cdots,\Controls_K)$ and write
\begin{equation}
  \Position = \SplineCoefs_t \Controls
\end{equation}
where
\begin{equation}
  \SplineCoefs_t =
  \begin{bmatrix}
    \SplineCoef_1(t) \Identity_{3 \cross 3} &
    \cdots &
    \SplineCoef_K(t) \Identity_{3 \cross 3}
  \end{bmatrix} ~.
\end{equation}
The spline parametrization is attractive for our purposes because the
position of the device is linear in the control points $\Controls$,
since $\SplineCoefs$ is a functionly only of time, which is
known. Furthermore, acceleration is also linear in the control points,
\begin{eqnarray}
  \LinearAccel &=& \SplineAccelCoefs_t \Controls \\
  \SplineAccelCoefs_t &=&
  \begin{bmatrix}
    \frac{d^2\SplineCoef_1(t)}{dt^2} \Identity_{3 \cross 3} &
    \cdots &
    \frac{d^2\SplineCoef_K(t)}{dt^2} \Identity_{3 \cross 3}
  \end{bmatrix}
\end{eqnarray}
The second derivatives $\frac{d^2\SplineCoef_K}{dt^2}$ are simple to
compute using the de Boor--Cox formula \cite{deboor1978}.

\subsection{Relative frame of reference}

We assume that estimates are available for the relative orientation of
the device over short time periods; that is, that we know $R_1^TR_2$
where $R_1$ and $R_2$ are the orientation of the device at any two
points within the time window being considered. Such relative
orientation estimates are easy to obtain by integrating angular
velocity measurements from a gyro as described in
\cite{StergiosOrientationFilter} or, if video frames are available at
the relevant times, by estimating the epipolar geometry for the two
images \cite{Zisserman}. We use the approach described in
\cite{Whatever}, which combines data from the gyro with epipolar
geometry constraints derived from feature correspondences.

By concatenating these relative orientations together, we may compute
the relative orientation between the device at the beginning of the
trajectory (time $t_0$) and its orientation at any other
time. Therefore, without loss of generality we henceforth define the
world frame to be the frame of reference of the device at $t_0$. The
positions and orientations of the other frames, as well as the
positions of the landmarks and the gravity vector will be expressed in
this frame.

\section{SOCP formulation for visual inertial navigation}

We now formulate the visual inertial navigation problem described in
the previous section as a second--order cone program. The variables we
wish to estimate are:
\begin{enumerate}
\item{The control points $\Controls$ parametrizing the trajectory.}
\item{The landmark coordinates $\Landmark_1,\cdots,\Landmark_M$}
\item{The accelerometer bias $\Bias$}
\item{The gravity vector $\Gravity$}
\end{enumerate}
Note that, as described above, we assume that relative orientation
estimates between the first frame and all other times are available,
so we do not directly estimate any rotation matrices. We do
\textit{not} assume that knowledge of the orientation of any frame
relative to gravity; this is captured by including the direction of
gravity as an unknown.

Given the sensor models described in the previous section, the maximum
likelihood estimate for all variables is obtained by solving
\begin{equation}
  \begin{aligned}
    & \underset{\AllVars}{\text{minimize}}
    & & \sum \| \hat{\Accel_i} - \Accel_i \|_{\AccelCov}^2 +
    \sum \| \hat{\Feature_i} - \Feature_i \|_{\FeatureCov}^2
  \end{aligned}
  \label{unconstrained-cost}
\end{equation}
where
\begin{eqnarray}
  \hat{\Accel_i} &=& R_i (\SplineAccelCoefs_i \Controls + \Gravity) + \Bias \\
  \hat{\Feature_{i}} &=& \Pr(K R_i (\Landmark_j - \SplineCoefs_i \Controls)) ~.
\end{eqnarray}
In the remainder of this section we re--cast this problem as an SOCP
of the form \eqnref{generic-socp}.

\subsection{Feature constraints}

While the first term in \eqnref{unconstrained-cost} is a linear least
squares cost and is therefore convex, the second term is non--convex
due to the non--linear projection function $\Pr(\cdot)$.  We therefore
replace this term with a set of constraints,
\begin{equation}
  \begin{aligned}
    & \underset{\AllVars}{\text{minimize}}
    & & \sum \| \hat{\Accel_i} - \Accel_i \|_{\AccelCov}^2 \\
    & \text{subject to}
    & & \| \hat{\Feature_i} - \Feature_i \| \leq \Tol
    ~~~~~~
    i = 1,\ldots,N
  \end{aligned}
  \label{constrained-cost}
\end{equation}
where $\Tol$ is a user--specified parameter, typically set to some
multiple of the expected feature noise. While this modified problem is
not precisely equivalent to \eqnref{unconstrained-cost}, it has the
distinct advantage of being globally convex, as we show
below. Furthermore, we show in the results section that solutions to
this modified problem closely approximate solutions to
\eqnref{unconstrained-cost} for real--world data.

We now define $\hat{\vec{y}_i} = K R_i (\Landmark_j - \SplineCoefs_i \Controls)$ and
expand the constraints in \eqnref{constrained-cost} as follows.
\begin{eqnarray}
  \| \hat{\Feature_i} - \Feature_i \| &\leq& \Tol \\
  \Bigl\| \frac{1}{\hat{\vec{y}_{i3}}}
  \begin{bmatrix}
    \hat{\vec{y}_{i1}} \\
      \hat{\vec{y}_{i2}}
  \end{bmatrix}
  - \Feature_i \Bigr\| &\leq& \Tol \\
  \Bigl\|
  \begin{bmatrix}
    \hat{\vec{y}_{i1}} \\
      \hat{\vec{y}_{i2}}
  \end{bmatrix} 
  - \Feature_i \hat{\vec{y}_{i3}} \Bigr\| &\leq& \Tol \hat{\vec{y}_{i3}}
\end{eqnarray}
Now, expanding
$K=\begin{bmatrix}\vec{k_1}&\vec{k_2}&\vec{k_3}\end{bmatrix}^T$ and
$\vec{\Feature_i}=\begin{bmatrix}\Feature_{i1}&\Feature_{i1}\end{bmatrix}^T$
we have
\begin{equation}
  \begin{split}
  \Bigl\|
  \begin{bmatrix}
    \vec{k_1}^T R_i (\Landmark_j - \SplineCoefs_i \Controls)
    - \Feature_{i1} \vec{k_3}^T R_i (\Landmark_j - \SplineCoefs_i \Controls)\\
    \vec{k_2}^T R_i (\Landmark_j - \SplineCoefs_i \Controls)
    - \Feature_{i2} \vec{k_3}^T R_i (\Landmark_j - \SplineCoefs_i \Controls)\\
  \end{bmatrix} 
  \Bigr\|\\
  \leq
  \Tol \vec{k_3}^T R_i (\Landmark_j - \SplineCoefs_i \Controls)
  \end{split}
\end{equation}
Finally we define
\begin{eqnarray*}
  Q_i &=& \begin{bmatrix}
    -\vec{k_1}^TR_i - \Feature_{i1} \vec{k_3}^T R_i &
    \Feature_{i1}\vec{k_3}^TR_i \SplineCoefs_i - \vec{k_1}^T R_i
    \SplineCoefs_i \\
    -\vec{k_2}^TR_i - \Feature_{i2} \vec{k_3}^T R_i &
    \Feature_{i2}\vec{k_3}^TR_i \SplineCoefs_i - \vec{k_2}^T R_i
    \SplineCoefs_i
  \end{bmatrix} \\
  \vec{s}_i &=& \begin{bmatrix}
    -\Tol \vec{k_3}^T R_i \SplineCoefs_i &
    \Tol \vec{k_3}^T R_i
  \end{bmatrix}^T
  \label{socp-constraints}
\end{eqnarray*}
and rewrite the original optimization problem as
\begin{equation}
  \begin{aligned}
    & \underset{\AllVars}{\text{minimize}}
    & & \sum \| \hat{\Accel_i} - \Accel_i \|_{\AccelCov}^2 \\
    & \text{subject to}
    & &
    \Bigl\| Q_i \begin{bmatrix}\Controls\\\Landmark_i\end{bmatrix} \Bigr\| \leq
    {\vec{s}_i}^T \begin{bmatrix}\Controls\\\Landmark_i\end{bmatrix}
    ~~~~~~
    i = 1,\ldots,N
  \end{aligned}
  \label{first-transform}
\end{equation}
Note that neither $Q_i$ nor $\vec{s}_i$ depend on any variables over
which we are optimizing, so the constraints above are second order
cone constraints of the form \eqnref{generic-socp}.

\subsection{Accelerometer Constraints}
Although the constraints in \eqnref{first-transform} have the desired
form, the objective in \eqnref{constrained-cost} is quadratic so does
not conform to the linear objective in \eqnref{generic-socp}. To
address this, we begin by expanding the objective,
\begin{eqnarray}
  \sum \| \hat{\Accel_i} - \Accel_i \|^2 &=&
  \sum \| R_i (\SplineAccelCoefs_i \Controls + \Gravity) + \Bias -
  \Accel_i \|^2\\
  &=& \| J \vec{w} - \vec{r} \|^2 \\
  &=& \vec{w}^T J^T J \vec{w} - 2 \vec{w}^T J^T \vec{r} + \vec{r}^T\vec{r}
\end{eqnarray}
where in the second line we defined
\begin{eqnarray}
  \vec{w} &=& \begin{bmatrix}\Controls&\Gravity&\Bias\end{bmatrix}^T \\
  \vec{r} &=& \begin{bmatrix} \Accel_1 & \cdots &
    \Accel_N \end{bmatrix}^T \\
  J &=& \begin{bmatrix}
    R_1\SplineAccelCoefs_1 & R_1 & \Identity \\
    \vdots & \vdots & \vdots \\
    R_N\SplineAccelCoefs_N & R_N & \Identity
  \end{bmatrix}
\end{eqnarray}
We now replace the quadratic objective in \eqnref{first-transform}
with a linear objective and a quadratic constraint,
\begin{equation}
  \begin{aligned}
    & \underset{\AllVars}{\text{minimize}} & & \lambda \\
    & \text{subject to}
    & & \vec{w}^T J^T J \vec{w} - 2 \vec{w}^T J^T \vec{r} +
    \vec{r}^T\vec{r} \leq \lambda \\
    & & &
    \Bigl\| Q_i \begin{bmatrix}\Controls\\\Landmark_i\end{bmatrix} \Bigr\| \leq
    {\vec{s}_i}^T \begin{bmatrix}\Controls\\\Landmark_i\end{bmatrix}
    ~~~~~~
    i = 1,\ldots,N
  \end{aligned}
  \label{second-transform}
\end{equation}
Finally, we rewrite this quadratic constraint as a second--order cone
constraint,
\begin{equation}
  \begin{aligned}
    & \underset{\AllVars}{\text{minimize}} & & \lambda \\
    & \text{subject to}
    & & \begin{split}
          \Bigl\| \begin{bmatrix}
            1 - 2\vec{w}^T J^T \vec{r} + \vec{r}^T\vec{r} - \lambda \\
            2 J \vec{w}
          \end{bmatrix} \Bigr\| \\
          \leq
          1 + 2 \vec{w}^T J^T \vec{r} - \vec{r}^T\vec{r} + \lambda
        \end{split}\\
    & & &
    \Bigl\| Q_i \begin{bmatrix}\Controls\\\Landmark_i\end{bmatrix} \Bigr\| \leq
    {\vec{s}_i}^T \begin{bmatrix}\Controls\\\Landmark_i\end{bmatrix}
    ~~~~~~
    i = 1,\ldots,N
  \end{aligned}
  \label{final-problem}
\end{equation}
Note again that neither $J$ nor $\vec{r}$ depend on the variables over
which we are optimizing, so \eqnref{final-problem} has the desired form
\eqnref{generic-socp}.
