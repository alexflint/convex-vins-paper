\documentclass[conference]{IEEEtran}
\usepackage{blindtext, graphicx}

% Add the compsoc option for Computer Society conferences.
%
% If IEEEtran.cls has not been installed into the LaTeX system files,
% manually specify the path to it like:
% \documentclass[conference]{../sty/IEEEtran}



% Some very useful LaTeX packages include:
% (uncomment the ones you want to load)


% *** MISC UTILITY PACKAGES ***
%
%\usepackage{ifpdf}
% Heiko Oberdiek's ifpdf.sty is very useful if you need conditional
% compilation based on whether the output is pdf or dvi.
% usage:
% \ifpdf
%   % pdf code
% \else
%   % dvi code
% \fi
% The latest version of ifpdf.sty can be obtained from:
% http://www.ctan.org/tex-archive/macros/latex/contrib/oberdiek/
% Also, note that IEEEtran.cls V1.7 and later provides a builtin
% \ifCLASSINFOpdf conditional that works the same way.
% When switching from latex to pdflatex and vice-versa, the compiler may
% have to be run twice to clear warning/error messages.






% *** CITATION PACKAGES ***
%
%\usepackage{cite}
% cite.sty was written by Donald Arseneau
% V1.6 and later of IEEEtran pre-defines the format of the cite.sty package
% \cite{} output to follow that of IEEE. Loading the cite package will
% result in citation numbers being automatically sorted and properly
% "compressed/ranged". e.g., [1], [9], [2], [7], [5], [6] without using
% cite.sty will become [1], [2], [5]--[7], [9] using cite.sty. cite.sty's
% \cite will automatically add leading space, if needed. Use cite.sty's
% noadjust option (cite.sty V3.8 and later) if you want to turn this off.
% cite.sty is already installed on most LaTeX systems. Be sure and use
% version 4.0 (2003-05-27) and later if using hyperref.sty. cite.sty does
% not currently provide for hyperlinked citations.
% The latest version can be obtained at:
% http://www.ctan.org/tex-archive/macros/latex/contrib/cite/
% The documentation is contained in the cite.sty file itself.






% *** GRAPHICS RELATED PACKAGES ***
%
\ifCLASSINFOpdf
  % \usepackage[pdftex]{graphicx}
  % declare the path(s) where your graphic files are
  % \graphicspath{{../pdf/}{../jpeg/}}
  % and their extensions so you won't have to specify these with
  % every instance of \includegraphics
  % \DeclareGraphicsExtensions{.pdf,.jpeg,.png}
\else
  % or other class option (dvipsone, dvipdf, if not using dvips). graphicx
  % will default to the driver specified in the system graphics.cfg if no
  % driver is specified.
  % \usepackage[dvips]{graphicx}
  % declare the path(s) where your graphic files are
  % \graphicspath{{../eps/}}
  % and their extensions so you won't have to specify these with
  % every instance of \includegraphics
  % \DeclareGraphicsExtensions{.eps}
\fi
% graphicx was written by David Carlisle and Sebastian Rahtz. It is
% required if you want graphics, photos, etc. graphicx.sty is already
% installed on most LaTeX systems. The latest version and documentation can
% be obtained at: 
% http://www.ctan.org/tex-archive/macros/latex/required/graphics/
% Another good source of documentation is "Using Imported Graphics in
% LaTeX2e" by Keith Reckdahl which can be found as epslatex.ps or
% epslatex.pdf at: http://www.ctan.org/tex-archive/info/
%
% latex, and pdflatex in dvi mode, support graphics in encapsulated
% postscript (.eps) format. pdflatex in pdf mode supports graphics
% in .pdf, .jpeg, .png and .mps (metapost) formats. Users should ensure
% that all non-photo figures use a vector format (.eps, .pdf, .mps) and
% not a bitmapped formats (.jpeg, .png). IEEE frowns on bitmapped formats
% which can result in "jaggedy"/blurry rendering of lines and letters as
% well as large increases in file sizes.
%
% You can find documentation about the pdfTeX application at:
% http://www.tug.org/applications/pdftex





% *** MATH PACKAGES ***
%
\usepackage[cmex10]{amsmath}
\usepackage{amssymb}
% A popular package from the American Mathematical Society that provides
% many useful and powerful commands for dealing with mathematics. If using
% it, be sure to load this package with the cmex10 option to ensure that
% only type 1 fonts will utilized at all point sizes. Without this option,
% it is possible that some math symbols, particularly those within
% footnotes, will be rendered in bitmap form which will result in a
% document that can not be IEEE Xplore compliant!
%
% Also, note that the amsmath package sets \interdisplaylinepenalty to 10000
% thus preventing page breaks from occurring within multiline equations. Use:
%\interdisplaylinepenalty=2500
% after loading amsmath to restore such page breaks as IEEEtran.cls normally
% does. amsmath.sty is already installed on most LaTeX systems. The latest
% version and documentation can be obtained at:
% http://www.ctan.org/tex-archive/macros/latex/required/amslatex/math/





% *** SPECIALIZED LIST PACKAGES ***
%
%\usepackage{algorithmic}
% algorithmic.sty was written by Peter Williams and Rogerio Brito.
% This package provides an algorithmic environment fo describing algorithms.
% You can use the algorithmic environment in-text or within a figure
% environment to provide for a floating algorithm. Do NOT use the algorithm
% floating environment provided by algorithm.sty (by the same authors) or
% algorithm2e.sty (by Christophe Fiorio) as IEEE does not use dedicated
% algorithm float types and packages that provide these will not provide
% correct IEEE style captions. The latest version and documentation of
% algorithmic.sty can be obtained at:
% http://www.ctan.org/tex-archive/macros/latex/contrib/algorithms/
% There is also a support site at:
% http://algorithms.berlios.de/index.html
% Also of interest may be the (relatively newer and more customizable)
% algorithmicx.sty package by Szasz Janos:
% http://www.ctan.org/tex-archive/macros/latex/contrib/algorithmicx/




% *** ALIGNMENT PACKAGES ***
%
%\usepackage{array}
% Frank Mittelbach's and David Carlisle's array.sty patches and improves
% the standard LaTeX2e array and tabular environments to provide better
% appearance and additional user controls. As the default LaTeX2e table
% generation code is lacking to the point of almost being broken with
% respect to the quality of the end results, all users are strongly
% advised to use an enhanced (at the very least that provided by array.sty)
% set of table tools. array.sty is already installed on most systems. The
% latest version and documentation can be obtained at:
% http://www.ctan.org/tex-archive/macros/latex/required/tools/


%\usepackage{mdwmath}
%\usepackage{mdwtab}
% Also highly recommended is Mark Wooding's extremely powerful MDW tools,
% especially mdwmath.sty and mdwtab.sty which are used to format equations
% and tables, respectively. The MDWtools set is already installed on most
% LaTeX systems. The lastest version and documentation is available at:
% http://www.ctan.org/tex-archive/macros/latex/contrib/mdwtools/


% IEEEtran contains the IEEEeqnarray family of commands that can be used to
% generate multiline equations as well as matrices, tables, etc., of high
% quality.


%\usepackage{eqparbox}
% Also of notable interest is Scott Pakin's eqparbox package for creating
% (automatically sized) equal width boxes - aka "natural width parboxes".
% Available at:
% http://www.ctan.org/tex-archive/macros/latex/contrib/eqparbox/





% *** SUBFIGURE PACKAGES ***
%\usepackage[tight,footnotesize]{subfigure}
% subfigure.sty was written by Steven Douglas Cochran. This package makes it
% easy to put subfigures in your figures. e.g., "Figure 1a and 1b". For IEEE
% work, it is a good idea to load it with the tight package option to reduce
% the amount of white space around the subfigures. subfigure.sty is already
% installed on most LaTeX systems. The latest version and documentation can
% be obtained at:
% http://www.ctan.org/tex-archive/obsolete/macros/latex/contrib/subfigure/
% subfigure.sty has been superceeded by subfig.sty.



%\usepackage[caption=false]{caption}
%\usepackage[font=footnotesize]{subfig}
% subfig.sty, also written by Steven Douglas Cochran, is the modern
% replacement for subfigure.sty. However, subfig.sty requires and
% automatically loads Axel Sommerfeldt's caption.sty which will override
% IEEEtran.cls handling of captions and this will result in nonIEEE style
% figure/table captions. To prevent this problem, be sure and preload
% caption.sty with its "caption=false" package option. This is will preserve
% IEEEtran.cls handing of captions. Version 1.3 (2005/06/28) and later 
% (recommended due to many improvements over 1.2) of subfig.sty supports
% the caption=false option directly:
%\usepackage[caption=false,font=footnotesize]{subfig}
%
% The latest version and documentation can be obtained at:
% http://www.ctan.org/tex-archive/macros/latex/contrib/subfig/
% The latest version and documentation of caption.sty can be obtained at:
% http://www.ctan.org/tex-archive/macros/latex/contrib/caption/




% *** FLOAT PACKAGES ***
%
%\usepackage{fixltx2e}
% fixltx2e, the successor to the earlier fix2col.sty, was written by
% Frank Mittelbach and David Carlisle. This package corrects a few problems
% in the LaTeX2e kernel, the most notable of which is that in current
% LaTeX2e releases, the ordering of single and double column floats is not
% guaranteed to be preserved. Thus, an unpatched LaTeX2e can allow a
% single column figure to be placed prior to an earlier double column
% figure. The latest version and documentation can be found at:
% http://www.ctan.org/tex-archive/macros/latex/base/



%\usepackage{stfloats}
% stfloats.sty was written by Sigitas Tolusis. This package gives LaTeX2e
% the ability to do double column floats at the bottom of the page as well
% as the top. (e.g., "\begin{figure*}[!b]" is not normally possible in
% LaTeX2e). It also provides a command:
%\fnbelowfloat
% to enable the placement of footnotes below bottom floats (the standard
% LaTeX2e kernel puts them above bottom floats). This is an invasive package
% which rewrites many portions of the LaTeX2e float routines. It may not work
% with other packages that modify the LaTeX2e float routines. The latest
% version and documentation can be obtained at:
% http://www.ctan.org/tex-archive/macros/latex/contrib/sttools/
% Documentation is contained in the stfloats.sty comments as well as in the
% presfull.pdf file. Do not use the stfloats baselinefloat ability as IEEE
% does not allow \baselineskip to stretch. Authors submitting work to the
% IEEE should note that IEEE rarely uses double column equations and
% that authors should try to avoid such use. Do not be tempted to use the
% cuted.sty or midfloat.sty packages (also by Sigitas Tolusis) as IEEE does
% not format its papers in such ways.





% *** PDF, URL AND HYPERLINK PACKAGES ***
%
%\usepackage{url}
% url.sty was written by Donald Arseneau. It provides better support for
% handling and breaking URLs. url.sty is already installed on most LaTeX
% systems. The latest version can be obtained at:
% http://www.ctan.org/tex-archive/macros/latex/contrib/misc/
% Read the url.sty source comments for usage information. Basically,
% \url{my_url_here}.


\usepackage{mydefs}



% *** Do not adjust lengths that control margins, column widths, etc. ***
% *** Do not use packages that alter fonts (such as pslatex).         ***
% There should be no need to do such things with IEEEtran.cls V1.6 and later.
% (Unless specifically asked to do so by the journal or conference you plan
% to submit to, of course. )


\begin{document}
%
% paper title
% can use linebreaks \\ within to get better formatting as desired
\title{A globally convex formulation for visual inertial navigation}


% author names and affiliations
% use a multiple column layout for up to three different
% affiliations
\author{
\IEEEauthorblockN{Alex Flint}
\IEEEauthorblockA{Independently funded\\
New York, NY, 10001\\
Email: alex.flint@gmail.com}
}

% conference papers do not typically use \thanks and this command
% is locked out in conference mode. If really needed, such as for
% the acknowledgment of grants, issue a \IEEEoverridecommandlockouts
% after \documentclass

% for over three affiliations, or if they all won't fit within the width
% of the page, use this alternative format:
% 
%\author{\IEEEauthorblockN{Michael Shell\IEEEauthorrefmark{1},
%Homer Simpson\IEEEauthorrefmark{2},
%James Kirk\IEEEauthorrefmark{3}, 
%Montgomery Scott\IEEEauthorrefmark{3} and
%Eldon Tyrell\IEEEauthorrefmark{4}}
%\IEEEauthorblockA{\IEEEauthorrefmark{1}School of Electrical and Computer Engineering\\
%Georgia Institute of Technology,
%Atlanta, Georgia 30332--0250\\ Email: see http://www.michaelshell.org/contact.html}
%\IEEEauthorblockA{\IEEEauthorrefmark{2}Twentieth Century Fox, Springfield, USA\\
%Email: homer@thesimpsons.com}
%\IEEEauthorblockA{\IEEEauthorrefmark{3}Starfleet Academy, San Francisco, California 96678-2391\\
%Telephone: (800) 555--1212, Fax: (888) 555--1212}
%\IEEEauthorblockA{\IEEEauthorrefmark{4}Tyrell Inc., 123 Replicant Street, Los Angeles, California 90210--4321}}




% use for special paper notices
%\IEEEspecialpapernotice{(Invited Paper)}




% make the title area
\maketitle


\begin{abstract}
%\boldmath
The visual inertial navigation problem involves estimating the trajectory of a moving device using measurements from visual and inertial sensors. Most previous work has applied least squares estimators (such as the EKF or the Levenberg--Marquardt algorithm) to cost functions that are not globally convex. But such approaches are not guaranteed to converge to any global optimum, and in practice local optima pose a real obstacle to robust performance. In this paper we present work towards a \textit{globally convex} formulation for visual inertial navigation, and describe an algorithm based on second--order cone programming that efficiently finds the unique global optimum. We demonstrate this algorithm reconstructing trajectories using measurements from standard cell--phone sensors.
\end{abstract}
% IEEEtran.cls defaults to using nonbold math in the Abstract.
% This preserves the distinction between vectors and scalars. However,
% if the journal you are submitting to favors bold math in the abstract,
% then you can use LaTeX's standard command \boldmath at the very start
% of the abstract to achieve this. Many IEEE journals frown on math
% in the abstract anyway.

% Note that keywords are not normally used for peerreview papers.
\begin{IEEEkeywords}
visual inertial odometry, convex optimization
\end{IEEEkeywords}



% For peer review papers, you can put extra information on the cover
% page as needed:
% \ifCLASSOPTIONpeerreview
% \begin{center} \bfseries EDICS Category: 3-BBND \end{center}
% \fi
%
% For peerreview papers, this IEEEtran command inserts a page break and
% creates the second title. It will be ignored for other modes.
\IEEEpeerreviewmaketitle

\section{Introduction}

Many applications require a mobile device to estimate its position and orientation over time. The Global Positioning System largely solves coarse--scale outdoor localization, but when sub--meter accuracy is required, or if operating in GPS--denied areas, such systems must use on--board sensors to estimate their position. In recent years there has been particular interest in visual inertial navigation (CITE), in which the available sensors are (1) a camera capturing video frames of the environment, (2) an accelerometer measuring the linear acceleration of the device\footnotemark, and (3) a gyro measuring the angular velocity of the device. This particular sensor combination is attractive because the sensors are small, cheap, require relatively little power, and are sufficient to reconstruct the metric trajectory of the device (CITE).

\footnotetext{real accelerometer measurements are distorted by gravity, bias, and noise terms, which will be described later.}

Two main approaches to visual inertial navigation have been proposed within the literature: filtering approaches, in which a covariance is tracked and updated, and optimization approaches, in which a cost is minimized at each time step. Early work within the robotics community pursued the former approach, mostly using the extended Kalman filter (CITE) and the closely related information filter (CITE). These approaches begin by linearizing the measurement equations relating state variables to sensor measurements, then proceed to update a covariance with each incoming sensor reading. Many extensions and improvements to the EKF have been described in recent years, some of which we survey in the next section.

In contrast, optimization--oriented approaches cast visual inertial navigation as an optimization problem in which the device trajectory is parametrized by some number of variables, and one minimizes a cost relating these variables to the sensor measurements (CITE). Having grown out of the computer vision literature on bundle adjustment, this literature almost universally uses quasi--Newton optimizers such as the Levenberg--Marquardt algorithm (CITE), the Dog--leg algorithm (CITE), or, in the case of very large--scale problems, conjugate gradient descent.

In fact both of these approaches can be understood as non--linear least--squares estimators. For globally convex problems, such estimators are guaranteed to converge to the global optimum in the limit of computation (CITE Boyd), but it is well--known that visual inertial navigation problems as commonly formulated are not globally convex, so the only guarantee one has is of eventual convergence to a local optima (CITE). This means that, depending on how an estimator is initialized, it may or may not recover the correct trajectory of the device, even in the limit of an infinite number of noise--free measurements, and even given the usual excitation requirements (CITE).

Furthermore, this is no mere theoretical technicality: practical visual inertial navigation systems are frequently hampered by failure to converge to the desired optimum, which is caused ultimately by the non--convexity of the problem (CITE). This issue is particularly severe during initialization, in which one seeks an estimate for the state of the device at time zero when no previous estimate is available to bootstrap from. The initialization problem has been recognized as a key unsolved problem in the field (CITE). In fact many state--of--the--art systems currently require a stationary period of several seconds at the beginning of a trajectory in order to overcome this issues (CITE), but as well as being inconvenient, such stipulations are impossible in passive contexts where one cannot request that the device undergo specific motion patterns.

In this paper we present a globally convex formulation for visual inertial navigation, allowing us to apply modern convex optimization tools with well--understood convergence guarantees. This allows us to overcome the initialization problem since at time zero we can formulate and solve a single convex optimization problem and be assured of convergence to the global optimum without requiring a previous estimate from which to bootstrap. Our work builds on ideas recently proposed within the computer vision literature where second--order cone programming (SOCP) has been applied to triangulation and resection problems (CITE).

Our approach is as follows. We begin by tracking image features across two or more video frames. Next, inspired by recent work on continuous--time SLAM (CITE), we parametrize the device trajectory as a cubic B--spline in three dimensions. Our optimization problem is then formulated over the spline control points, plus the inertial biases, the direction of gravity, and the coordinates for $N$ unknown landmarks corresponding to tracked image features. The continuous--time trajectory parametrization allows us to write closed--form expressions for the predicted acceleration and angular velocity measurements as a function of the spline control points, as well as the image projections of the estimated landmarks. Next we define a residual vector consisting of the difference between the expected and observed sensor readings, for both the visual and inertial sensors. While least--squares algorithms would minimize the $L_2$ norm of this vector, we opt instead to minimize the $L_\infty$ norm. This means we are no longer minimizing the true negative log--likelihood, but the fact that we can minimize this cost globally makes this concession bearable. Furthermore, we show in the experimental section that in practice our minimization results in an excellent approximation to the $L_2$ minimizer. Finally we solve the $L_\infty$ minimization as a sequence of SOCP problems.

The remainder of this paper is organized as follows. The next section surveys existing work related to our own, then section 3 presents our convex formulation for the visual inertial navigation problem. Section 4 describes the SOCP optimization algorithm that we use to solve this problem, then section 5 presents experimental results using real and simulated data. Finally, section 6 concludes the paper.


\section{Related work}

There is a long history within the robotics community of applying the EKF to navigation problems. Early work by Smith and Cheeseman \cite{smith1986} described an approach that maintains a joint covariance over the robot position and the coordinates of surrounding landmarks. Davison \etal \cite{davison2003} described a SLAM filter based on the EKF that uses only a monocular camera. The variable--state dimension filter described by McLauchlan \cite{mclauchlan1999} provided a framework for changing the size of the state space over time. Jones and Soatto \cite{jones2011} describe an extended SLAM filter incorporating visual and inertial measurements.

A second approach to filtering is to leave the landmarks out of the filter state, which can significantly reduce computation time since the EKF is cubic in the size of the filter state, and landmarks typically account for many state variables. Soatto \etal \cite{soatto1996} used the epipolar geometry in conjunction with inertial constraints to formulate such a filter. Garcia \etal \cite{garcia2002} and Mourikis and Roumeliotis \cite{mourikis2007} described sliding window filters in which the state space consists of the device state at multiple points in time. This permits optimal visual updates without explicitly tracking a covariance over landmarks.

In contrast, the vision community has explored approaches in which navigation is cast as an optimization problem. The visual odometry algorithm of Nister \etal \cite{nister2004} solves a sequence of two-- and three--frame optimization problems using a pre--emptive RANSAC algorithm. Sibley \cite{sibley2006} described a sliding window bundle adjuster in which navigation is solved as a sequence of non--linear least squares problems. Klein and Murray \cite{klein2007} perform global bundle adjustment on a small number of frames sampled from the video stream. Strasdat \etal \cite{strasdat2011} describe a windowing approach in which spatially distant frames are included in the optimization but not actually adjusted in the update step.

All of the above approaches use a discrete--time representation in which trajectories are parameterized by the position and orientation of the device at the time video frames were captured. Recently, Furgale \etal \cite{furgale2012} presented a continuous--time formulation in which the trajectory of the device is parametrized via the control points of a spatial B--spline. Lovegrove \etal extended this approach to incorporate unknown landmarks in the optimization. A key advantage of the continuous--time approach is that it removes the need to solve an ODE when incorporating inertial measurements; instead, both the vision and inertial measurements can be incorporated into a straightforward generative model. This is crucial to the work presented in this paper since solutions of the relevant ODEs are not analytic and so are ill--suited to the convex optimization algorithms we use.

Second--order cone programming has a deep history in the convex optimization literature; an overview is given in \cite{boyd2008}. Within the computer vision literature, Kahl \cite{kahl2005} formulated and solved a number of classic reconstruction problems including triangulation, resectioning, and homography estimation as SOCP problems. A related quasi--convex optimization scheme was described by Ke \etal \cite{ke2007}. This paper builds on these ideas to formulate visual inertial navigation as a SOCP problem, and we show how to estimate not just discrete camera poses but full device trajectories, incorporating constraints based on both visual and inertial measurements.

\section{Second order cone programming}

A convex program is an optimization problem of the form
\begin{equation}
    \begin{aligned}
    & \text{minimize} & & f(\Landmark) \\
    & \text{subject to} & & h_i(\Landmark) \geq 0 ~~~~~~ i = 1,\ldots,N
  \end{aligned}
\end{equation}
where $f$ is a convex function and $\{\Landmark : h_i(\Landmark) \geq 0 ~\forall i\}$
is a convex set. All convex programs can be solved \cite{boyd}, but
certain problem classes permit particularly efficient solutions. One
such class is the second--order cone program (SOCP), which has the
form
\begin{equation}
  \begin{aligned}
    & \text{minimize} & & \vec{f}^T \Landmark & \\
    & \text{subject to} & & 
    \|A_i\Landmark+\vec{b}_i\|_2 \leq \vec{c}_i^T\Landmark + d_i & i =
    1,\ldots,N \\
    & & & G_j\Landmark = \vec{h}_j & j = 1,\ldots,M
  \end{aligned}
  \label{generic-socp}
\end{equation}
SOCP problems can be solved efficiently using primal--dual interior
point methods \cite{boyd} and several off--the--shelf solvers are
available \cite{mosek,cvxopt,cvx}.

\section{Problem Statement}

We wish to estimate the trajectory of a mobile device given three
types of sensor measurements: accelerometer readings, gyro readings,
and image features. These quantities will be estimated with respect to
an arbitrary (not necessarily gravity--aligned) frame of reference we
will call the world frame. We now describe models for each sensor.

Accelerometer measurements are sampled according to
\begin{equation}
  \Accel = R (\LinearAccel + \Gravity) + \Bias + \AccelNoise
\end{equation}
where $R$ is the orientation of the device in the world frame,
$\LinearAccel$ is the acceleration of the device in the world frame,
$\Gravity$ is the gravity vector in the world frame, $\Bias$ is an
additive bias term, and $\AccelNoise$ is a Gaussian white noise term
with covariance $\AccelCov$. Whereas we assume that the bias $\Bias$
is fixed for all sensor measurements \footnotemark, the noise term is
$\AccelNoise$ is sampled independently for each measurement.

\footnotetext{this is a reasonable assumption for MEMS sensors
as found on modern cell phones for trajectories of up to several minutes}

Gyro measurements are sampled according to
\begin{equation}
  \Gyro = R \AngularVelocity + \GyroBias + \GyroNoise
\end{equation}
where $\AngularVelocity$ is the angular velocity of the device in the
world frame, $\GyroBias$ is an additive bias, and $\GyroNoise$ is a
Gaussian white noise term with covariance $\GyroCov$.

Finally, image measurements are sampled according to
\begin{equation}
  \Feature = \Pr(K R (\Landmark - \Position)) + \FeatureNoise
\end{equation}
where $R$ and $p$ are respectively the orientation and position of the
device in the world frame at the time the feature was, $K$ is the
camera intrinsics, $\Landmark$ is the world coordinates of the
landmark giving rise to the feature, and $\FeatureNoise$ is a Gaussian
white noise term with covariance $\FeatureCov$. $\Pr$ is the
projection function,
\begin{equation}
  \Pr\Bigl(\begin{bmatrix}x\\y\\z\end{bmatrix}\Bigr) =
    \frac{1}{z}\begin{bmatrix}x\\y\end{bmatrix}
\end{equation}

\subsection{Trajectory representation}

We choose to parametrize the trajectory of the device as a cubic
B--spline, as described in \cite{furgale,lovegrove}. Given
control points $\Control_1,\cdots,\Control_K$, $\Control_i \in
\Reals$, the position of the device at time $t$ is,
\begin{equation}
  \Position = \sum \SplineCoef_i(t) \Control_i
\end{equation}
where $\SplineCoef_1,\cdots,\SplineCoef_K$ are the spline basis
functions, which depend only on time and can be computed using the de
Boor--Cox formula \cite{deboor1978}. For brevity, we concatenate the
control points into a vector
$\Controls=(\Controls_1,\cdots,\Controls_K)$ and write
\begin{equation}
  \Position = \SplineCoefs_t \Controls
\end{equation}
where
\begin{equation}
  \SplineCoefs_t =
  \begin{bmatrix}
    \SplineCoef_1(t) \Identity_{3 \cross 3} &
    \cdots &
    \SplineCoef_K(t) \Identity_{3 \cross 3}
  \end{bmatrix} ~.
\end{equation}
The spline parametrization is attractive for our purposes because the
position of the device is linear in the control points $\Controls$,
since $\SplineCoefs$ is a functionly only of time, which is
known. Furthermore, acceleration is also linear in the control points,
\begin{eqnarray}
  \LinearAccel &=& \SplineAccelCoefs_t \Controls \\
  \SplineAccelCoefs_t &=&
  \begin{bmatrix}
    \frac{d^2\SplineCoef_1(t)}{dt^2} \Identity_{3 \cross 3} &
    \cdots &
    \frac{d^2\SplineCoef_K(t)}{dt^2} \Identity_{3 \cross 3}
  \end{bmatrix}
\end{eqnarray}
The second derivatives $\frac{d^2\SplineCoef_K}{dt^2}$ are simple to
compute using the de Boor--Cox formula \cite{deboor1978}.

\subsection{Relative frame of reference}

We assume that estimates are available for the relative orientation of
the device over short time periods; that is, that we know $R_1^TR_2$
where $R_1$ and $R_2$ are the orientation of the device at any two
points within the time window being considered. Such relative
orientation estimates are easy to obtain by integrating angular
velocity measurements from a gyro as described in
\cite{StergiosOrientationFilter} or, if video frames are available at
the relevant times, by estimating the epipolar geometry for the two
images \cite{Zisserman}. We use the approach described in
\cite{Whatever}, which combines data from the gyro with epipolar
geometry constraints derived from feature correspondences.

By concatenating these relative orientations together, we may compute
the relative orientation between the device at the beginning of the
trajectory (time $t_0$) and its orientation at any other
time. Therefore, without loss of generality we henceforth define the
world frame to be the frame of reference of the device at $t_0$. The
positions and orientations of the other frames, as well as the
positions of the landmarks and the gravity vector will be expressed in
this frame.

\section{SOCP formulation for visual inertial navigation}

We now formulate the visual inertial navigation problem described in
the previous section as a second--order cone program. The variables we
wish to estimate are:
\begin{enumerate}
\item{The control points $\Controls$ parametrizing the trajectory.}
\item{The landmark coordinates $\Landmark_1,\cdots,\Landmark_M$}
\item{The accelerometer bias $\Bias$}
\item{The gravity vector $\Gravity$}
\end{enumerate}
Note that, as described above, we assume that relative orientation
estimates between the first frame and all other times are available,
so we do not directly estimate any rotation matrices. We do
\textit{not} assume that knowledge of the orientation of any frame
relative to gravity; this is captured by including the direction of
gravity as an unknown.

Given the sensor models described in the previous section, the maximum
likelihood estimate for all variables is obtained by solving
\begin{equation}
  \begin{aligned}
    & \underset{\AllVars}{\text{minimize}}
    & & \sum \| \hat{\Accel_i} - \Accel_i \|_{\AccelCov}^2 +
    \sum \| \hat{\Feature_i} - \Feature_i \|_{\FeatureCov}^2
  \end{aligned}
  \label{unconstrained-cost}
\end{equation}
where
\begin{eqnarray}
  \hat{\Accel_i} &=& R_i (\SplineAccelCoefs_i \Controls + \Gravity) + \Bias \\
  \hat{\Feature_{i}} &=& \Pr(K R_i (\Landmark_j - \SplineCoefs_i \Controls)) ~.
\end{eqnarray}
In the remainder of this section we re--cast this problem as an SOCP
of the form \eqnref{generic-socp}.

\subsection{Feature constraints}

While the first term in \eqnref{unconstrained-cost} is a linear least
squares cost and is therefore convex, the second term is non--convex
due to the non--linear projection function $\Pr(\cdot)$.  We therefore
replace this term with a set of constraints,
\begin{equation}
  \begin{aligned}
    & \underset{\AllVars}{\text{minimize}}
    & & \sum \| \hat{\Accel_i} - \Accel_i \|_{\AccelCov}^2 \\
    & \text{subject to}
    & & \| \hat{\Feature_i} - \Feature_i \| \leq \Tol
    ~~~~~~
    i = 1,\ldots,N
  \end{aligned}
  \label{constrained-cost}
\end{equation}
where $\Tol$ is a user--specified parameter, typically set to some
multiple of the expected feature noise. While this modified problem is
not precisely equivalent to \eqnref{unconstrained-cost}, it has the
distinct advantage of being globally convex, as we show
below. Furthermore, we show in the results section that solutions to
this modified problem closely approximate solutions to
\eqnref{unconstrained-cost} for real--world data.

We now define $\hat{\vec{y}_i} = K R_i (\Landmark_j - \SplineCoefs_i \Controls)$ and
expand the constraints in \eqnref{constrained-cost} as follows.
\begin{eqnarray}
  \| \hat{\Feature_i} - \Feature_i \| &\leq& \Tol \\
  \Bigl\| \frac{1}{\hat{\vec{y}_{i3}}}
  \begin{bmatrix}
    \hat{\vec{y}_{i1}} \\
      \hat{\vec{y}_{i2}}
  \end{bmatrix}
  - \Feature_i \Bigr\| &\leq& \Tol \\
  \Bigl\|
  \begin{bmatrix}
    \hat{\vec{y}_{i1}} \\
      \hat{\vec{y}_{i2}}
  \end{bmatrix} 
  - \Feature_i \hat{\vec{y}_{i3}} \Bigr\| &\leq& \Tol \hat{\vec{y}_{i3}}
\end{eqnarray}
Now, expanding
$K=\begin{bmatrix}\vec{k_1}&\vec{k_2}&\vec{k_3}\end{bmatrix}^T$ and
$\vec{\Feature_i}=\begin{bmatrix}\Feature_{i1}&\Feature_{i1}\end{bmatrix}^T$
we have
\begin{equation}
  \begin{split}
  \Bigl\|
  \begin{bmatrix}
    \vec{k_1}^T R_i (\Landmark_j - \SplineCoefs_i \Controls)
    - \Feature_{i1} \vec{k_3}^T R_i (\Landmark_j - \SplineCoefs_i \Controls)\\
    \vec{k_2}^T R_i (\Landmark_j - \SplineCoefs_i \Controls)
    - \Feature_{i2} \vec{k_3}^T R_i (\Landmark_j - \SplineCoefs_i \Controls)\\
  \end{bmatrix} 
  \Bigr\|\\
  \leq
  \Tol \vec{k_3}^T R_i (\Landmark_j - \SplineCoefs_i \Controls)
  \end{split}
\end{equation}
Finally we define
\begin{eqnarray*}
  Q_i &=& \begin{bmatrix}
    -\vec{k_1}^TR_i - \Feature_{i1} \vec{k_3}^T R_i &
    \Feature_{i1}\vec{k_3}^TR_i \SplineCoefs_i - \vec{k_1}^T R_i
    \SplineCoefs_i \\
    -\vec{k_2}^TR_i - \Feature_{i2} \vec{k_3}^T R_i &
    \Feature_{i2}\vec{k_3}^TR_i \SplineCoefs_i - \vec{k_2}^T R_i
    \SplineCoefs_i
  \end{bmatrix} \\
  \vec{s}_i &=& \begin{bmatrix}
    -\Tol \vec{k_3}^T R_i \SplineCoefs_i &
    \Tol \vec{k_3}^T R_i
  \end{bmatrix}^T
  \label{socp-constraints}
\end{eqnarray*}
and rewrite the original optimization problem as
\begin{equation}
  \begin{aligned}
    & \underset{\AllVars}{\text{minimize}}
    & & \sum \| \hat{\Accel_i} - \Accel_i \|_{\AccelCov}^2 \\
    & \text{subject to}
    & &
    \Bigl\| Q_i \begin{bmatrix}\Controls\\\Landmark_i\end{bmatrix} \Bigr\| \leq
    {\vec{s}_i}^T \begin{bmatrix}\Controls\\\Landmark_i\end{bmatrix}
    ~~~~~~
    i = 1,\ldots,N
  \end{aligned}
  \label{first-transform}
\end{equation}
Note that neither $Q_i$ nor $\vec{s}_i$ depend on any variables over
which we are optimizing, so the constraints above are second order
cone constraints of the form \eqnref{generic-socp}.

\subsection{Accelerometer Constraints}
Although the constraints in \eqnref{first-transform} have the desired
form, the objective in \eqnref{constrained-cost} is quadratic so does
not conform to the linear objective in \eqnref{generic-socp}. To
address this, we begin by expanding the objective,
\begin{eqnarray}
  \sum \| \hat{\Accel_i} - \Accel_i \|^2 &=&
  \sum \| R_i (\SplineAccelCoefs_i \Controls + \Gravity) + \Bias -
  \Accel_i \|^2\\
  &=& \| J \vec{w} - \vec{r} \|^2 \\
  &=& \vec{w}^T J^T J \vec{w} - 2 \vec{w}^T J^T \vec{r} + \vec{r}^T\vec{r}
\end{eqnarray}
where in the second line we defined
\begin{eqnarray}
  \vec{w} &=& \begin{bmatrix}\Controls&\Gravity&\Bias\end{bmatrix}^T \\
  \vec{r} &=& \begin{bmatrix} \Accel_1 & \cdots &
    \Accel_N \end{bmatrix}^T \\
  J &=& \begin{bmatrix}
    R_1\SplineAccelCoefs_1 & R_1 & \Identity \\
    \vdots & \vdots & \vdots \\
    R_N\SplineAccelCoefs_N & R_N & \Identity
  \end{bmatrix}
\end{eqnarray}
We now replace the quadratic objective in \eqnref{first-transform}
with a linear objective and a quadratic constraint,
\begin{equation}
  \begin{aligned}
    & \underset{\AllVars}{\text{minimize}} & & \lambda \\
    & \text{subject to}
    & & \vec{w}^T J^T J \vec{w} - 2 \vec{w}^T J^T \vec{r} +
    \vec{r}^T\vec{r} \leq \lambda \\
    & & &
    \Bigl\| Q_i \begin{bmatrix}\Controls\\\Landmark_i\end{bmatrix} \Bigr\| \leq
    {\vec{s}_i}^T \begin{bmatrix}\Controls\\\Landmark_i\end{bmatrix}
    ~~~~~~
    i = 1,\ldots,N
  \end{aligned}
  \label{second-transform}
\end{equation}
Finally, we rewrite this quadratic constraint as a second--order cone
constraint,
\begin{equation}
  \begin{aligned}
    & \underset{\AllVars}{\text{minimize}} & & \lambda \\
    & \text{subject to}
    & & \begin{split}
          \Bigl\| \begin{bmatrix}
            1 - 2\vec{w}^T J^T \vec{r} + \vec{r}^T\vec{r} - \lambda \\
            2 J \vec{w}
          \end{bmatrix} \Bigr\| \\
          \leq
          1 + 2 \vec{w}^T J^T \vec{r} - \vec{r}^T\vec{r} + \lambda
        \end{split}\\
    & & &
    \Bigl\| Q_i \begin{bmatrix}\Controls\\\Landmark_i\end{bmatrix} \Bigr\| \leq
    {\vec{s}_i}^T \begin{bmatrix}\Controls\\\Landmark_i\end{bmatrix}
    ~~~~~~
    i = 1,\ldots,N
  \end{aligned}
  \label{final-problem}
\end{equation}
Note again that neither $J$ nor $\vec{r}$ depend on the variables over
which we are optimizing, so \eqnref{final-problem} has the desired form
\eqnref{generic-socp}.

\section{Results}

Simulation results, with comparison to Mourikis. Could also compare to

- optimize from zero

- stationary initialization

- linear solution for camera positions without using IMU terms at all

- robust vs non-robust version

Rolling shutter results with/without rolling shutter model?

Results on a small number of real-world datasets. The right thing to look at here is how many converged vs how many did not converge.


\section{Conclusion}

Emphasize first application of modern convex programming to visual inertial navigation. Emphasize that the spline formulation is important.









% needed in second column of first page if using \IEEEpubid
%\IEEEpubidadjcol

% An example of a floating figure using the graphicx package.
% Note that \label must occur AFTER (or within) \caption.
% For figures, \caption should occur after the \includegraphics.
% Note that IEEEtran v1.7 and later has special internal code that
% is designed to preserve the operation of \label within \caption
% even when the captionsoff option is in effect. However, because
% of issues like this, it may be the safest practice to put all your
% \label just after \caption rather than within \caption{}.
%
% Reminder: the "draftcls" or "draftclsnofoot", not "draft", class
% option should be used if it is desired that the figures are to be
% displayed while in draft mode.
%
%\begin{figure}[!t]
%\centering
%\includegraphics[width=2.5in]{myfigure}
% where an .eps filename suffix will be assumed under latex, 
% and a .pdf suffix will be assumed for pdflatex; or what has been declared
% via \DeclareGraphicsExtensions.
%\caption{Simulation Results}
%\label{fig_sim}
%\end{figure}

% Note that IEEE typically puts floats only at the top, even when this
% results in a large percentage of a column being occupied by floats.


% An example of a double column floating figure using two subfigures.
% (The subfig.sty package must be loaded for this to work.)
% The subfigure \label commands are set within each subfloat command, the
% \label for the overall figure must come after \caption.
% \hfil must be used as a separator to get equal spacing.
% The subfigure.sty package works much the same way, except \subfigure is
% used instead of \subfloat.
%
%\begin{figure*}[!t]
%\centerline{\subfloat[Case I]\includegraphics[width=2.5in]{subfigcase1}%
%\label{fig_first_case}}
%\hfil
%\subfloat[Case II]{\includegraphics[width=2.5in]{subfigcase2}%
%\label{fig_second_case}}}
%\caption{Simulation results}
%\label{fig_sim}
%\end{figure*}
%
% Note that often IEEE papers with subfigures do not employ subfigure
% captions (using the optional argument to \subfloat), but instead will
% reference/describe all of them (a), (b), etc., within the main caption.


% An example of a floating table. Note that, for IEEE style tables, the 
% \caption command should come BEFORE the table. Table text will default to
% \footnotesize as IEEE normally uses this smaller font for tables.
% The \label must come after \caption as always.
%
%\begin{table}[!t]
%% increase table row spacing, adjust to taste
%\renewcommand{\arraystretch}{1.3}
% if using array.sty, it might be a good idea to tweak the value of
% \extrarowheight as needed to properly center the text within the cells
%\caption{An Example of a Table}
%\label{table_example}
%\centering
%% Some packages, such as MDW tools, offer better commands for making tables
%% than the plain LaTeX2e tabular which is used here.
%\begin{tabular}{|c||c|}
%\hline
%One & Two\\
%\hline
%Three & Four\\
%\hline
%\end{tabular}
%\end{table}


% Note that IEEE does not put floats in the very first column - or typically
% anywhere on the first page for that matter. Also, in-text middle ("here")
% positioning is not used. Most IEEE journals use top floats exclusively.
% Note that, LaTeX2e, unlike IEEE journals, places footnotes above bottom
% floats. This can be corrected via the \fnbelowfloat command of the
% stfloats package.





% if have a single appendix:
%\appendix[Proof of the Zonklar Equations]
% or
%\appendix  % for no appendix heading
% do not use \section anymore after \appendix, only \section*
% is possibly needed

% use appendices with more than one appendix
% then use \section to start each appendix
% you must declare a \section before using any
% \subsection or using \label (\appendices by itself
% starts a section numbered zero.)
%


% use section* for acknowledgement
\section*{Acknowledgment}


The authors would like to thank...


% Can use something like this to put references on a page
% by themselves when using endfloat and the captionsoff option.
\ifCLASSOPTIONcaptionsoff
  \newpage
\fi



% trigger a \newpage just before the given reference
% number - used to balance the columns on the last page
% adjust value as needed - may need to be readjusted if
% the document is modified later
%\IEEEtriggeratref{8}
% The "triggered" command can be changed if desired:
%\IEEEtriggercmd{\enlargethispage{-5in}}

% references section

% can use a bibliography generated by BibTeX as a .bbl file
% BibTeX documentation can be easily obtained at:
% http://www.ctan.org/tex-archive/biblio/bibtex/contrib/doc/
% The IEEEtran BibTeX style support page is at:
% http://www.michaelshell.org/tex/ieeetran/bibtex/
%\bibliographystyle{IEEEtran}
% argument is your BibTeX string definitions and bibliography database(s)
%\bibliography{IEEEabrv,../bib/paper}
%
% <OR> manually copy in the resultant .bbl file
% set second argument of \begin to the number of references
% (used to reserve space for the reference number labels box)
\bibliographystyle{IEEEtran}
\bibliography{references}
%\thebibliography

% biography section
% 
% If you have an EPS/PDF photo (graphicx package needed) extra braces are
% needed around the contents of the optional argument to biography to prevent
% the LaTeX parser from getting confused when it sees the complicated
% \includegraphics command within an optional argument. (You could create
% your own custom macro containing the \includegraphics command to make things
% simpler here.)
%\begin{biography}[{\includegraphics[width=1in,height=1.25in,clip,keepaspectratio]{mshell}}]{Michael Shell}
% or if you just want to reserve a space for a photo:

\begin{IEEEbiography}[{\includegraphics[width=1in,height=1.25in,clip,keepaspectratio]{picture}}]{John Doe}
\blindtext
\end{IEEEbiography}

% You can push biographies down or up by placing
% a \vfill before or after them. The appropriate
% use of \vfill depends on what kind of text is
% on the last page and whether or not the columns
% are being equalized.

%\vfill

% Can be used to pull up biographies so that the bottom of the last one
% is flush with the other column.
%\enlargethispage{-5in}

\end{document}


